\chapter*{Introduction}
\addcontentsline{toc}{chapter}{Introduction}

Major goal of this thesis is to retroactively add support for distributed
computation to an existing and widespread language with functional features.

\section{Distributing Computation}

Erlang, Elixir. But no types. Go for fault tolerance? Bloom language.

Sequoia. (?)

Oz dataflow.

Datalog, Datomic, Cascalog. Query evaluation with Datalog is based on first
order logic, and is thus sound and complete. However, Datalog is not Turing
complete, and is thus used as a domain-specific language that can take advantage
of efficient algorithms developed for query resolution. Datalog is a lightweight
deductive database system where queries and database updates are expressed in
the logic language. The use of Datalog syntax and an implementation based on
tabling intermediate results ensures that all queries terminate.

Datalog is a nonprocedural query language based on the logic-programming
language Prolog.  A user describes the information desired without giving a
specific procedure for obtaining that information. Datalog simplifies writing

Much effort invested in a formal foundation for concurrent and distributed
programming -- e.g., process calculi. Pi calculus. Join calculus.

Join-patterns provides a way to write concurrent, parallel and distributed
computer programs by message passing. Compared to the use of threads and locks,
this is a high level programming model using communication constructs model to
abstract the complexity of concurrent environment and to allow scalability. Its
focus is on the execution of a chord between messages atomically consumed from a
group of channels.

Recent research has explored using Datalog-based languages to express a
distributed system as a set of logical invariants.

Languages foundations like Dedalus~\cite{Dedalus} focus on providing a minimal set of
primitives for programming and reasoning about distributed systems. Dedalus
reduces to a subset of Datalog with negation, aggregate functions,
successor and choice, and admits an explicit representation of time into the
logic language in order to provide a declarative foundation for the two
signature features of distributed systems: mutable state, and asynchronous
processing and communication.

NoSQL = Datalog! Function-passing is a stateless model of this idea. All about
using logic to reduce coordination.

These languages and efforts differ from the effort presented in this thesis in
that logic-based . They are trying to figure out ways to jointly do
optimizations and to ensure a number of other properties important to
distributed systems with state remain in tact. When a theoretical result is
arrived at in this work of databases and data-centric programming systems, the
sole effort to date concerned with integrating these advancements into the
programming language has focused on providing a restricted logic programming
language to attempt to ensure correctness and ease of reasoning. These are not
the goals of this thesis. Instead, we focus less on providing langauge featuers designed to shield users from complications that arise when having to worry about consistency (after all,


The rationale of some in purporting a language to be more adept at distributed
programming has focused on languages which remove limitations and which enable
more features to be distributed.

That is not the central tenet of this thesis. Rather, this thesis argues that
while indeed more language features ought to be more reliably distributed (e.g.,
function closures) it might not be preferable that all features be given
unrestricted support for distribution. That is, in this thesis,

This thesis follows the advice given in 's Note On Distributed computing, where
important advice is to avoid abstracting over network communication. Other
models tout such abstraction to be a feature, or to from a programming language
perspective provide for a more powerful or otherwise desirable . Like Note, we
argue that this is not the case. Designers of distributed systems must have a
handle on and be aware of all situations that could cause network communication.

The distributed model of computation behind Oz~\cite{DistributedOz} abstracts
over communication boundaries. Network transparency
~\cite{ConceptsTechniquesModelsProgramming}

In this thesis, rather than abstracting over network communication, we provide
firmer, more desirable primitives to initiate your own network communication.




\section{Contributions}

\section{Structure}

\lipsum[1]
