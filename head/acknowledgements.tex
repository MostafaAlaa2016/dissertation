\chapter*{Acknowledgements}
\markboth{Acknowledgements}{Acknowledgements}
\addcontentsline{toc}{chapter}{Acknowledgements}

\vspace{0.5cm}
A PhD is never easy for anyone. For most, it's a road fraught with challenges,
technical and ideological. I had a rougher start than most, bouncing around
completely disparate fields for two full years, an entire ocean away from home,
in a country where I knew no one and couldn't speak the local language before I
joined the LAMP group. Therefore, I must first and foremost thank my advisor,
Martin Odersky, for looking at this oddball PhD student with a background in
signal processing and electrical engineering, in another research group, doing
something totally different, and giving me a chance to try and build meaningful
frameworks and abstractions as part of the Scala team at EPFL these past four
years. Without his support and insight, this dissertation would not have been
possible.

\vspace{0.5cm}
I'd also like to thank my friends in Lausanne and colleagues at EPFL, those in
LAMP and those not. If it wasn't for you, I'd not have gotten here. Switzerland
can be a lonely place for those from far away. You were the people I could speak
to, joke with, and generally relax around during these long six years. The list
is long, and I hope I manage to mention everybody.

\vspace{0.5cm}
To my friends who started this journey with me; roommates and EDIC office
colleagues, thank you. Yuliy Schwarzburg, a longtime friend from Cooper Union in
New York City, and roommate here in Switzerland, and his fianc\'{e}e Lyvia
Fishman. Arash Farhang, my best mountain buddy and now caretaker of my old best
friend, Umlaut. Evan Williams and Davide de Masi -- 'MURICAH! -- thanks for all
of the good times and solidarity in being ignorant Americans lost on this
continent without HVAC and diners together. Petr Susil, Iulian Dragos, Tanja
Petricevic, Cristina Ghiurcuta, Jennifer Sartor, Eva Darulova, Tihomir Gvero,
Horesh Ben Shitrit, Adar Hoffman, and Alla Merzakreeva, thank you for being some
of my first friends in Switzerland.

\vspace{0.5cm}
One person that stood out during these years in Switzerland is Liz Daley. Liz
didn't live in any one place. She rode big mountains and climbed epic splitter
all over the world, based often in Seattle or Chamonix/Lausanne. She lived her
dreams and became one of the first and few pro woman snowboarders and
mountaineers. I never had the chance to tell her how inspiring she was. Liz, you
constantly remind me what stoke is. Even I (one of a thousand distant
non-mountaineering buddies) think of you often. You were an example to myself
and many. Your time with us was far too short.

\vspace{0.5cm}
Importantly, I want to thank my colleagues in the LAMP laboratory. You
all were the source of so many deep discussions, explorations of ideas, and of
course many beers, ski trips, or other unforgettable shenanigans. Sandro Stucki,
Manohar Jonnalagedda, Alex Prokopec, Ingo Maier, Vojin Jovanovic, Hubert
Plociniczak, Tobias Schlatter, Donna Malayeri, Vlad Ureche, Lukas Rytz, Adriaan
Moors, Gilles Dubochet, Tiark Rompf, Miguel Garcia, Denys Shabalin, Eugene
Burmako, S\'{e}bastien Doeraene, Christopher Jan Vogt, Dmitry Petrashko,
Samuel Gr\"{u}tter, Nada Amin, and Antonio Cunei -- thank you for the
camaraderie all these years. And of course, thank you to Danielle Chamberlain
and Fabien Salvi for fielding all of my administrative and computer-related
questions, respectively.

\vspace{0.5cm}
Being in a position to start a PhD is one thing, and a PhD thesis
acknowledgement section wouldn't be complete without thanking those unknowing
mentors who are largely responsible for me taking this path to higher education
at all. Firstly, I must thank a high school teacher of mine, all the way back to
the days where I majored in fine arts at Alexander W. Dreyfoos School of the
Arts back home in West Palm Beach, Florida. Jenny Gifford helped me to realize
that I had any potential at all. Without her encouragement, I'd have never ended
up at univeristy at all, let alone a top university like Cooper Union. So Jenny,
thank you for seeing something in me. Without you, I'd not be where I am.
Secondly, I'd like to thank Professor Bethanie Stadler, a short-term advisor I
had while I participated in a US National Science Research Experience for
Undergrads (REU) program at the University of Minnesota. Professor Stadler is a
professor of Materials Science -- a field I knew nothing about upon starting an
REU with her. Professor Stadler helped me to realize that I was capable of doing
independent research, even if I was entering a new field where I had little to
no experience at the onset. The encouragement she showed me during my time in
Minnesota led to a trip to present our work at a scientific conference in the
French Alps -- my first trip to Europe, and the turning point that helped me to
realize (1) I can do a PhD, and (2) in Switzerland. Beth introduced me to EPFL.
Without her, I'd likely not have gone to grad school, and I'd never have heard
of EPFL.

\vspace{0.5cm}
I would like to thank my close friends here in Switzerland and back home in the
US. Darja Jovanavic and (not-so) little David Jovanovic, you were my buddies in
Lausanne through all of the moments when life and the PhD got tough -- a mere
thank you is not enough. I'd also like to thank my lifelong friends back home in
the US, Lindsay Hebrank, Beth Bachelor, thanks for always being a friend, no
matter how far apart we are.

\vspace{0.5cm}
I'd also like to thank my siblings for teaching me so much about life and even
kids. Thank you to my little sisters Ashley Marcantonio and Kayla Marcantonio
for always teaching me something new and for always such being a riot. And of
course, thank you to my little brother Sean Miller for teaching me the virtue of
patience.

\vspace{0.5cm}
I'd like to thank my parents, my mother Christina Ellis Marcantonio, my
step-father Timothy Marcantonio, and my father Steve Miller for being there for
me all these years. I know we didn't come from a lot, but you did the best for
me that you could, and I will always be thankful. Mom -- if you hadn't patiently
spent your days teaching me how to read and write as a toddler, I might not have
ever realized the power of knowledge and thought. You gave me the educational
foundation that I have built upon throughout my entire life, and for that I am
eternally grateful.

\vspace{0.5cm}
I want to thank my husband Daniel Klug for his everlasting patience and
unconditional love and support. Daniel has had to put up with many late nights
spanning from from paper deadlines, to lectures, to organizing conferences, as
well as months of me traveling, from India to San Francisco, and all the while,
he has been there for me, helping me through life with an uplifiting smile, and
always a hilarious pun. I'm truly lucky to have you by my side in this life, and
I still don't know what I did to deserve you.

\vspace{0.5cm}
Last but not least, I'd like to thank Philipp Haller, my closest friend these
past five years, and my co-author. Philipp, you were the one that stood next to
me through all of the toughest moments and greatest triumphs during the PhD
these past five years. I will never forget what we went through together -- the
paper pushes, the epic travels, the times when life in general got immeasurably
difficult. Not only did you stand with me through the hard times and the good,
always kind, forgiving, and patient, but you also spent countless hours teaching
me much of what I know about Computer Science and Programming Languages. I know
I can never repay the time that you invested in me, and for that, know that I am
forever grateful. Thank you from the bottom of my heart.

\bigskip

\noindent\textit{Basel, Switzerland, July 26th, 2015}
\hfill H.~M.
